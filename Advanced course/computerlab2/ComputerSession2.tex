%2multibyte Version: 5.50.0.2960 CodePage: 1252

\documentclass{article}
%%%%%%%%%%%%%%%%%%%%%%%%%%%%%%%%%%%%%%%%%%%%%%%%%%%%%%%%%%%%%%%%%%%%%%%%%%%%%%%%%%%%%%%%%%%%%%%%%%%%%%%%%%%%%%%%%%%%%%%%%%%%%%%%%%%%%%%%%%%%%%%%%%%%%%%%%%%%%%%%%%%%%%%%%%%%%%%%%%%%%%%%%%%%%%%%%%%%%%%%%%%%%%%%%%%%%%%%%%%%%%%%%%%%%%%%%%%%%%%%%%%%%%%%%%%%
\usepackage{amsmath}

\setcounter{MaxMatrixCols}{10}
%TCIDATA{OutputFilter=LATEX.DLL}
%TCIDATA{Version=5.50.0.2960}
%TCIDATA{Codepage=1252}
%TCIDATA{<META NAME="SaveForMode" CONTENT="1">}
%TCIDATA{BibliographyScheme=Manual}
%TCIDATA{Created=Tuesday, June 13, 2017 09:19:44}
%TCIDATA{LastRevised=Friday, May 18, 2018 15:52:38}
%TCIDATA{<META NAME="GraphicsSave" CONTENT="32">}
%TCIDATA{<META NAME="DocumentShell" CONTENT="Standard LaTeX\Blank - Standard LaTeX Article">}
%TCIDATA{Language=American English}
%TCIDATA{CSTFile=40 LaTeX article.cst}

\newtheorem{theorem}{Theorem}
\newtheorem{acknowledgement}[theorem]{Acknowledgement}
\newtheorem{algorithm}[theorem]{Algorithm}
\newtheorem{axiom}[theorem]{Axiom}
\newtheorem{case}[theorem]{Case}
\newtheorem{claim}[theorem]{Claim}
\newtheorem{conclusion}[theorem]{Conclusion}
\newtheorem{condition}[theorem]{Condition}
\newtheorem{conjecture}[theorem]{Conjecture}
\newtheorem{corollary}[theorem]{Corollary}
\newtheorem{criterion}[theorem]{Criterion}
\newtheorem{definition}[theorem]{Definition}
\newtheorem{example}[theorem]{Example}
\newtheorem{exercise}[theorem]{Exercise}
\newtheorem{lemma}[theorem]{Lemma}
\newtheorem{notation}[theorem]{Notation}
\newtheorem{problem}[theorem]{Problem}
\newtheorem{proposition}[theorem]{Proposition}
\newtheorem{remark}[theorem]{Remark}
\newtheorem{solution}[theorem]{Solution}
\newtheorem{summary}[theorem]{Summary}
\newenvironment{proof}[1][Proof]{\noindent\textbf{#1.} }{\ \rule{0.5em}{0.5em}}
\input{tcilatex}
\begin{document}


\begin{center}
\textbf{Computer Tutorial 2: Hierarchical Priors for Bayesian VARs}
\end{center}

In this exercise sheet, I\ provide a series of questions and answers
relating to hierarchical priors in VARs. Exercise 1 is taken from the new
edition of Bayesian Econometric Methods and includes both theoretical
derivations and an empirical application. As with Computer Tutorial 1, I\ am
not expecting you to go through the theoretical derivations and proofs in
detail. Rather, the computer session will focus on the computational part of
the exercise (part b). Code which does these exercises in provided. In the
computer lab, I suggest you experiment with these codes (e.g. try different
priors, lag lengths, etc.)\ to familiarize yourself with Bayesian
programming in VARs.

\newpage

\textbf{Exercise 1 (Stochastic Search Variable Selection in VARs)}

SSVS\ methods are a potentially useful way of surmounting
over-parameterization and over-fitting concerns which arise when the number
of potential explanatory variables is large relative to the number of
observations. VARs, where each of $n$ equations contains $p$ lags of each of 
$n$ dependent variables, also have many potential explanatory variables. And
the error covariance matrix, $\Sigma $, contains $\frac{n\left( n-1\right) }{%
2}$ free elements. Thus, with large VARs, involving dozens or more dependent
variables, there are a huge number of parameters to estimate. Concerns about
over-parameterization and over-fitting are likely to be especially severe
with VARs. In light of these concerns, you are asked to:

(a) Derive a Gibbs sampling algorithm which allows for Bayesian estimation
of the VAR\ using a SSVS prior (VAR-SSVS)

(b) Use the data set from Computer Tutorial 1 estimate the VAR-SSVS. Examine
how sensitive your findings are to prior hyperparameter choices.

\textbf{Solution to Exercise 1}

(a) This solution sets up the prior and derives the posterior conditionals
in the same manner as George, Sun and Ni (2008).

We use notation where $\beta $ is the $K\times 1$ vector of VAR\
coefficients and $\Sigma $ is the $n\times n$ error covariance matrix. We
work with the Cholesky decomposition:

\begin{equation*}
\Sigma ^{-1}=\Psi \Psi ^{\prime }
\end{equation*}%
where $\Psi $ is upper-triangular with elements $\psi _{ij}$. The diagonal
elements of the latter matrix are denoted by $\psi =\left( \psi
_{11},..,\psi _{nn}\right) ^{\prime }$ and the off-diagonal upper-triangular
elements by $\eta =\left( \eta _{2}^{\prime },..,\eta _{n}^{\prime }\right)
^{\prime }$ where $\eta _{j}=\left( \psi _{1j},..,\psi _{j-1,j}\right)
^{\prime }$ for $j=2,..,n$.

The model is parameterized in terms of $\beta ,\psi $ and $\eta $. SSVS can
be interpreted as defining a particular hierarchical prior for these
parameters. In particular, the prior for the VAR\ coefficients is%
\begin{equation}
\beta |\gamma ^{\beta }\sim N\left( 0,\underline{D}^{\beta }\right)
\label{SSVS_prior_beta}
\end{equation}%
where $\underline{D}^{\beta }$ is a diagonal matrix with diagonal elements $%
\underline{d}_{i}^{\beta }$ for $i=1,..,K$ and $\gamma ^{\beta }=\left(
\gamma _{1}^{\beta },..,\gamma _{k}^{\beta }\right) ^{\prime }$. The SSVS\
nature of this prior arises by having $\gamma _{i}^{\beta }\in \left\{
0,1\right\} $ and defining:%
\begin{equation*}
\underline{d}_{i}^{\beta }=\left\{ 
\begin{tabular}{l}
$\tau _{0i}^{2}$ if $\gamma _{i}^{\beta }=0$ \\ 
$\tau _{1i}^{2}$ if $\gamma _{i}^{\beta }=1$%
\end{tabular}%
\right.
\end{equation*}%
where $\tau _{0i}^{2}$ is small and $\tau _{1i}^{2}$ is large.\footnote{%
George, Sun and Ni (2008) describe a semi-default automatic approach to
objectively choosing these small and large prior variances. In the empirical
illustration of part b), we subjectively choose them.} Thus, if $\gamma
_{i}^{\beta }=0$ the corresponding VAR\ coefficient is shrunk to be close to
zero, whereas if $\gamma _{i}^{\beta }=1$ it is estimated in a relatively
unrestricted fashion. $\gamma ^{\beta }$ is a vector of unknown parameters
and, thus, requires a prior. We assume the elements of $\gamma ^{\beta }$ to
be, a priori, independent of one another with%
\begin{equation}
\Pr \left( \gamma _{i}^{\beta }=0\right) =\underline{p}_{i}
\label{SSVS_prior_gammab}
\end{equation}%
and, thus, $\Pr \left( \gamma _{i}^{\beta }=1\right) =1-\underline{p}_{i}$

The hierarchical prior for $\eta $ has the same SSVS form allowing for
individual error covariances to be shrunk towards zero or estimated in an
unrestricted fashion. In particular, we assume, for $j=2,..,n$,%
\begin{equation}
\eta _{j}|\gamma _{j}^{\eta }\sim N\left( 0,\underline{D}_{j}^{\eta }\right)
\label{SSVS_prior_eta}
\end{equation}%
where $\underline{D}_{j}^{\eta }=diag\left( \underline{d}_{1j}^{\eta },..,%
\underline{d}_{j-1,j}^{\eta }\right) $. The vector of SSVS\ indicator
variables, $\gamma _{ij}^{\eta }\in \left\{ 0,1\right\} $, is defined as $%
\gamma ^{\eta }=\left( \gamma _{2}^{\eta },..,\gamma _{n}^{\eta }\right)
^{\prime }$ where $\gamma _{j}^{\eta }=\left( \gamma _{1j}^{\eta },..,\gamma
_{j-1,j}^{\eta }\right) ^{\prime }$. Next we assume%
\begin{equation*}
\underline{d}_{ij}^{\eta }=\left\{ 
\begin{tabular}{l}
$\kappa _{0ij}^{2}$ if $\gamma _{ij}^{\eta }=0$ \\ 
$\kappa _{1ij}^{2}$ if $\gamma _{ij}^{\eta }=1$%
\end{tabular}%
\right.
\end{equation*}%
where $\kappa _{0ij}^{2}$ is small and $\kappa _{1i}^{2}$ is large. We
assume the elements of $\gamma ^{\eta }$ to be, a priori, independent of one
another with%
\begin{equation}
\Pr \left( \gamma _{ij}^{\eta }=0\right) =\underline{q}_{ij}.
\label{SSVS_prior_gammaa}
\end{equation}

Finally, the error variances are assumed to have Gamma priors (independent
of one another) and, thus,%
\begin{equation}
\psi _{ii}^{2}\sim G\left( \underline{a}_{i},\underline{b}_{i}\right)
\label{SSVS_prior_psi}
\end{equation}%
for $i=1,..,n$.

We next derive a Gibbs sampler involving the posterior conditionals for $%
\beta ,\psi ,$ $\eta ,$ $\gamma ^{\beta }$ and $\gamma ^{\eta }$.

We do not provide details of the posterior conditional for $\beta $ since
they are given in the solution to Exercise 1 of Computer Session 1 with
prior hyperparameters $\beta _{0},V_{\boldsymbol{\beta }}$ of that exercise
replaced by $0,\underline{D}^{\beta }$ from (\ref{SSVS_prior_beta}). That
is, conditional on $\gamma ^{\beta }$, the SSVS prior for $\beta $ is a
Normal prior of exactly the same form as in yesterday's exercise.

Any Bayesian posterior involves multiplying likelihood times prior. To get a
conditional posterior for a specific parameter, one fixes all the other
parameters and treats the expression as a p.d.f for the specific parameter.
If one follows this strategy for the posterior of $\gamma ^{\beta }$
conditional on all the other parameters things are greatly simplified by
noting that $\gamma ^{\beta }$ does not enter the likelihood. In fact it
only appears in (\ref{SSVS_prior_beta}) and (\ref{SSVS_prior_gammab}).
Multiplying the Normal form for the former by the Bernoulli form for the
latter yields a conditional posterior of a simple form. To be precise, the
assumption that $\gamma _{i}^{\beta }$ is, a priori, independent of $\gamma
_{j}^{\beta }$ for $i\neq j$ means that we can draw the SSVS variable
selection indicators independently of one another with probabilities (for $%
i=1,..,k$) 
\begin{equation*}
\Pr \left( \gamma _{i}^{\beta }=0|y,\beta ,\eta ,\psi \right) =\overline{p}%
_{i}
\end{equation*}%
where 
\begin{equation*}
\overline{p}_{i}=\frac{\tau _{0i}^{-1}\exp \left( -\frac{\beta _{i}^{2}}{%
2\tau _{0i}^{2}}\right) \underline{p}_{i}}{\tau _{0i}^{-1}\exp \left( -\frac{%
\beta _{i}^{2}}{2\tau _{0i}^{2}}\right) \underline{p}_{i}+\tau
_{1i}^{-1}\exp \left( -\frac{\beta _{i}^{2}}{2\tau _{1i}^{2}}\right) \left(
1-\underline{p}_{i}\right) }
\end{equation*}%
where the denominator of this expression ensures probabilities sum to one.
Furthermore, $\Pr \left( \gamma _{i}^{\beta }=1|y,\beta ,\eta ,\psi \right)
=1-\overline{p}_{i}$.

To obtain the conditional posterior for $\eta $, we use notation from the
lecture slides and write the VAR\ in matrix form and let 
\begin{equation*}
S=\left( Y-XA\right) \left( Y-XA\right) ^{\prime }
\end{equation*}%
be the $n\times n$ sum of squares matrix of the VAR\ errors with upper-left $%
j\times j$ sub-matrix being denoted by $S_{j}$ and individual elements of $S$
being $s_{ij}$ and $s_{j}=\left( s_{1j},..,s_{j-1,j}\right) ^{\prime }$.
Using the properties of the matric-variate Normal distribution, the
likelihood function is proportional to%
\begin{equation*}
\left\vert \Psi \right\vert ^{T}\exp \left\{ tr\left( -\frac{1}{2}\Psi
^{\prime }S\Psi \right) \right\} .
\end{equation*}%
Since $\Psi $ is upper-triangular we have $\left\vert \Psi \right\vert
^{T}=\dprod\limits_{i=1}^{n}\psi _{ii}^{T}$. And, multiplying out the terms
inside the trace operator we have%
\begin{equation*}
\sum_{i=1}^{n}\psi _{ii}^{2}v_{i}+\sum_{j=2}^{n}\left( \eta _{j}+\psi
_{jj}S_{j-1}^{-1}s_{j}\right) ^{\prime }S_{j-1}\left( \eta _{j}+\psi
_{jj}S_{j-1}^{-1}s_{j}\right) .
\end{equation*}%
Putting these two pieces together we obtain a likelihood proportional to%
\begin{equation}
\dprod\limits_{i=1}^{n}\psi _{ii}^{T}\exp \left\{ -\frac{1}{2}\left[
\sum_{i=1}^{n}\psi _{ii}^{2}v_{i}+\sum_{j=2}^{n}\left( \eta _{j}+\psi
_{jj}S_{j-1}^{-1}s_{j}\right) ^{\prime }S_{j-1}\left( \eta _{j}+\psi
_{jj}S_{j-1}^{-1}s_{j}\right) \right] \right\} .  \label{VAR_like}
\end{equation}%
Note that the likelihood breaks into independent components for each $\eta
_{j}$ for $j=2,..,p$ and each has a Normal form. The priors given by (\ref%
{SSVS_prior_eta}) are also Normal and independent of one another. It is
straightforward to multiply each Normal prior by the Normal likelihood
component for $\eta _{j}$ to obtain for $j=2,..,p$ 
\begin{equation*}
\eta _{j}|y,\beta ,\psi \sim N\left( \overline{\mu }_{j},\overline{D}%
_{j}^{\eta }\right)
\end{equation*}%
where%
\begin{equation*}
\overline{D}_{j}^{\eta }=\left[ S_{j-1}+\left( \underline{D}_{j}^{\eta
}\right) ^{-1}\right] ^{-1}
\end{equation*}%
\begin{equation*}
\overline{\mu }_{j}=-\psi _{jj}\overline{D}_{j}^{\eta }s_{j}.
\end{equation*}

To obtain the posterior conditionals for $\gamma ^{\eta }$ we follow the
same strategy (and almost identical derivations)\ as for $\gamma ^{\beta }$
and find (for $j=2,..,n$ and $i=1..,j-1$)%
\begin{equation*}
\Pr \left( \gamma _{ij}^{\eta }=0|y,\beta ,\eta ,\psi \right) =\overline{q}%
_{ij}
\end{equation*}%
where 
\begin{equation*}
\overline{q}_{ij}=\frac{\kappa _{0ij}^{-1}\exp \left( -\frac{\psi _{ij}^{2}}{%
2\kappa _{0ij}^{2}}\right) \underline{q}_{ij}}{\kappa _{0ij}^{-1}\exp \left(
-\frac{\psi _{ij}^{2}}{2\kappa _{0ij}^{2}}\right) \underline{q}_{i}+\kappa
_{1ij}^{-1}\exp \left( -\frac{\psi _{ij}^{2}}{2\kappa _{1ij}^{2}}\right)
\left( 1-\underline{q}_{ij}\right) }
\end{equation*}%
where the denominator of this expression ensures probabilities sum to one.
Furthermore, $\Pr \left( \gamma _{ij}^{\eta }=1|y,\beta ,\eta ,\psi \right)
=1-\overline{q}_{ij}$.

Finally, to obtain draws of $\psi $, this can be done by using the
independent Gamma conditional posteriors for $\psi _{jj}^{2}$ for $j=1,..,n$
which are:%
\begin{equation*}
\psi _{ii}^{2}|y,\beta ,\gamma ^{\beta }\sim G\left( \overline{a}_{i},%
\overline{b}_{i}\right)
\end{equation*}%
where 
\begin{eqnarray*}
\overline{a}_{i} &=&\underline{a}_{i}+\frac{1}{2} \\
\overline{b}_{1} &=&\underline{b}_{1}+\frac{s_{11}}{2} \\
\overline{b}_{j} &=&\underline{b}_{j}+\frac{1}{2}\left\{
s_{jj}+s_{j}^{\prime }\left[ S_{j-1}+\left( \underline{D}_{j}^{\eta }\right)
^{-1}\right] ^{-1}s_{j}\right\} \text{ for }j=2,..,n.
\end{eqnarray*}

The proof is standard since it involves multiplying the Gamma prior (\ref%
{SSVS_prior_psi}) by the Normal likelihood in (\ref{VAR_like}) and
collecting terms involving $\psi _{ii}^{2}$.

(b) Matlab code (named VAR\_SSVS.m) which answers this question is available
on this book's website. The results below set $p=1$ and include an
intercept. Prior hyperparameters are set as $\underline{q}_{ij}=\underline{p}%
_{i}=\frac{1}{2},$ $\tau _{0i}=\kappa _{0ij}=0.01$, $\tau _{1i}=\kappa
_{1ij}=100$ $\ \underline{a}_{i}=\underline{b}_{i}=0.01$ for all $i$ and $j$
(although you may wish to experiment with other prior hyperparameter values).

The following table, based on 22,000 draws of which the first 2,000 were
discarded as the burn-in, presents posterior means, standard deviations and
the posterior inclusion probability (i.e. $\Pr \left( \gamma _{i}^{\beta
}=1|y\right) $ for each $i=1,..,K$). It can be seen that posterior inclusion
probabilities are very high for the first own lags in each of the three
equations and their coefficients are similar to the comparable Minnesota
prior results in Table 1. However, relative to Table 1, the remainder of the
coefficients have been shrunk towards zero.

\begin{center}
\begin{tabular}{|l|l|l|l|}
\hline
\multicolumn{4}{|l|}{Table 1:\ Posterior Results for VAR\ Coefficients using
VAR-SSVS} \\ \hline
& Mean & St. Deviation & Inclusion Probability \\ \hline
& \multicolumn{3}{|l|}{Inflation Equation} \\ \hline
Lag inflation & $0.793$ & $0.049$ & $1.000$ \\ \hline
Lag unemployment & $0.010$ & $0.007$ & $3\times 10^{-4}$ \\ \hline
Lag interest rate & $0.020$ & $0.009$ & $0.015$ \\ \hline
& \multicolumn{3}{|l|}{Unemployment Equation} \\ \hline
Lag inflation & $0.093$ & $0.074$ & $0.595$ \\ \hline
Lag unemployment & $0.969$ & $0.007$ & $1.000$ \\ \hline
Lag interest rate & $0.014$ & $0.010$ & $0.003$ \\ \hline
& \multicolumn{3}{|l|}{Interest Rate Equation} \\ \hline
Lag inflation & $0.139$ & $0.207$ & $0.329$ \\ \hline
Lag unemployment & $0.003$ & $0.009$ & $1\times 10^{-4}$ \\ \hline
Lag interest rate & $0.965$ & $0.035$ & $1.000$ \\ \hline
\end{tabular}
\end{center}

Table 2 is of the same format as Table 1, except it contains posterior
results for $\Sigma .$ The SSVS algorithm does not apply to the error
variances, only the covariances which is why there are no inclusion
probabilities to be reported for the former. It can be seen that there is
one covariance, $\Sigma _{21}$, that is being shrunk to zero. The VAR-SSVS
is providing strong evidence that $\Sigma _{31}$ and $\Sigma _{32}$ are
non-zero indicating that the error in the interest rate equation is
correlated with the errors of the other two equations.

\begin{center}
\begin{tabular}{|l|l|l|l|}
\hline
\multicolumn{4}{|l|}{Table 2:\ Posterior Results for $\Sigma $ using VAR-SSVS
} \\ \hline
& Mean & St. Deviation & Inclusion Probability \\ \hline
$\Sigma _{11}$ & $0.177$ & $0.018$ & n.a. \\ \hline
$\Sigma _{21}$ & $-0.001$ & $0.005$ & $0.053$ \\ \hline
$\Sigma _{31}$ & $0.112$ & $0.028$ & $1.000$ \\ \hline
$\Sigma _{22}$ & $0.089$ & $0.009$ & n.a. \\ \hline
$\Sigma _{32}$ & $-0.142$ & $0.023$ & $1.000$ \\ \hline
$\Sigma _{33}$ & $0.910$ & $0.090$ & n.a. \\ \hline
\end{tabular}

\newpage 
\end{center}

\textbf{Exercise 2 (Dirichlet Laplace Hierarchical Priors in VARs)}

The Dirichlet Laplace hierarchical prior was covered in the lecture and I\
will not provide more details of the statistical theory underlying it. If
you want to learn more about the theory, the following paper is an excellent
source:

Bhattacharya, A., Pati, D., Pillai, N. S., \& Dunson, D. B. (2015).
Dirichlet--Laplace priors for optimal shrinkage. Journal of the American
Statistical Association, 110(512), 1479-1490.

If you want to see a good empirical application which uses this approach in
VARs, then the following paper is a good one:

Kastner, G. and Huber, F. (2017). Sparse Bayesian vector autoregressions in
huge dimensions available at https://arxiv.org/abs/1704.03239

I provide code (BVAR\_DL.m) which estimates this model on a small data set
for you to experiment with. 

\bigskip

\end{document}
